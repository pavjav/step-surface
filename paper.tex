\documentclass[]{article}
\usepackage{amsfonts}
\usepackage{amsmath}
\usepackage{graphicx}
\usepackage{amsthm}

\newtheorem{thm}{Theorem}[subsection]


%opening
\title{Dynamics of Geodesic Flows on the Infinite Cube Step Surface}
\author{Pavel Javornik}

\begin{document}

\maketitle

\begin{abstract}

\end{abstract}

\section{Introduction}
\subsection{Background}
The purpose of the following paper is to study the non-compact, infinite surface generated entirely out of the edge-to-edge gluing of square faces, whose final product resembles cubes stacked in a cascading, pyramidal step pattern (or perhaps a rhombile tiling). (See Fig //) The surface we call the "Cube Step" is a member of the family of infinite Euclidean cone surfaces built out of rectangles with vertices of cone angle $3\pi$ and $\frac{3\pi}{2}$. We wish to study the surface in the case where we glue only squares with edges of length 1. Specifically, we want to study geodesic flows on this surface and to prove that properties of the initial trajectory angle determine whether or not the geodesic will close or drift periodically. 

The determining factor is simply whether or not the trajectory angle vector direction $(a,b)$, relative to the bottom and left edges of any square face, has integer values $a$,$b$ (where $a$ and $b$ are relatively prime) are of the same parity (odd-odd slope) or different parities (even-odd slope). We choose to neglect cases where paths are incident with vertices, and proceed to treat them as singularities. With these conditions in place we found that any geodesic with an initial odd-odd trajectory closes on the surface, while the even-odd cases drift periodically, tending to infinity.

To better understand this surface, we will adapt methods used in the study of Veech surfaces to suit our needs. These methods have become incredibly valuable to anyone studying rational billiard flows on a variety of compact or (in the case of infinite surfaces) non-compact surfaces. Geodesic flows in these problems are characterized as point masses moving along frictionless surfaces that experience perfectly elastic collisions with the boundaries of the surface. The straight line path is then deformed by a reflection over the boundary in the opposite direction. Translation surfaces constructed in the following manner therefore take on symmetric properties in accordance to these transformations and the types of boundaries the point mass might encounter. That is to say that orienting these flows requires that the translation surface itself be reflected to realign the flow. These translation surfaces are often described as being "unfolded" by some billiard flow, or strung along a laundry line of polygons congruent to the base. In the later sections we will describe how our surface's canonical form deforms geodesic flows by rotation and how we can use these rotations and edge-crossings to determine whether or not a geodesic will close.

What inspired this paper was the work done by Hubert et al. [Hu] on the Ehrenfest-Wind Tree model. Any readers who are familiar with their paper or the Tree model will most likely notice the similarities between the quotient spaces of the two surfaces and the properties that determine whether or not geodesic paths close. While similar, these surfaces are not entirely identical. The effect that a boundary collision has on a geodesic alters the dynamical properties of the surface. Or perhaps they better demonstrate properties of the surface itself. Another fundamental difference between the two is that the Cube Step is a surface with cone angles surrounding an infinite number of singularities on the surface. The problem has to take into careful consideration what it means for a path to cross a saddle connection between these singularities of the translation surface.



\subsection{Procedure}
The Cube Step surface has an interesting construction process that lets us take each face of the cube and arrange them in any way we choose so long as we do not change how the edges are identified, or glued together. It is preferable to study the flows on this surface as though every cube has been flattened onto the plane. This means that we can consider the cube step to be tiled by infinitely many of L-shaped surfaces whose edges are identified by rotation.

Doing so greatly reduces the number of possible directions that a flow might travel. Since every edge of this L-shaped surface is identified to another either by translation or a 90 degree rotation, any straight line path on this flattened version of the Cube Step can experience only one of four possible directions at any given time, whereas the Cube Step would've had considerably more.

Moreover, every face of this L-shaped surface is given by rotating its associated face on the Cube Step about the x or y axis, affine translations, or a combination of the two. By cutting and gluing every half-cube of the Cube Step in a consistent fashion, we can preserve the areas of the faces and distances between any two points on the surface with the flattening procedure. We describe this procedure as piecewise maps that are comprised of compositions of Euclidean matrices and translations of points relative to the L-shape onto a subset of $\mathbb{R}^{2}$ with square shaped holes cut out. Since rotations and translations of points preserves distance between them, the path lengths and  associated properties of geodesic flows remain invariant.

The image of this map is a flattened version of the Cube Step. (See FIG [MATLAB MODEL]). From that surface we construct a new one made out of four copies of original, each rotated by some factor of 90 degrees so as to realign the geodesic flows and assure that they can only travel in one possible direction. This four-fold surface is then a cover of the original surface. We consider every possible direction to be a "phase" of the flow, and index the planes accordingly. Any edges in the holes of the surface are identified with an edge on an adjacent phase plane (By rotation of $\pm \frac{\pi}{4}$). Any point on this surface has a coordinate in $\mathbb{R}\times\mathbb{R}\times\mathbb{Z}_{4} $

Similarly, we take the quotient space of the flat Cube Step centered around each square shaped hole and realign its flows by rotation. The resulting surface is a compact branch cover of the quotient space. Singularities on the translation surface have a total angle that is some multiple of $2\pi$. We consider the translation surface to be a base surface of the four-fold cover, and study its properties to prove our main theorem. In particular, we wish to determine what effect a member of the translation surface's Veech group has on both the surface and geodesic flow in hopes that it would shine some light on odd-odd even-odd phenomenon.



 $SL_{2}(\mathbb{R})$

\section{Constructing the Surface}
This section will detail how the Cube Step is constructed out of an infinite number of squares, and explain the process by which we flatten the structure in such a way that preserves the metric, and makes it simpler for us to describe the behavior of geodesics at these edges.

\subsection{The Cube Step Surface and Isometric Variant }
To build the surface out of infinitely many squares, we take the union of the instances of three squares, each lying on planes orthogonal to the other two and sharing an edge with exactly one other square. The three squares have one vertex in common.

Every instance can be described as cube whose bottom, left, and rear faces have been removed. They will be arranged in such a way that every integer triple $m,n,p\in\mathbb{Z}$ will have this surface associated with it. To each face we ascribe a letter A, B, or C. This will be useful when defining the flattening map.
\begin{gather*}
\chi_{m,n,p} = ([m, m+1]\times[n,n+1]\times\{p\})
\\\cup  (\{m+1\}\times[n,n+1]\times[p,p-1]) 
\\\cup ([m,m+1]\times\{n+1\}\times[p,p-1])
\end{gather*}

is the family of these instances based at some integer triple. The surface will be comprised of the union of the members that satisfy the property, $m+n+p=0$. From here on we will refer to the Cube Step surface embedded form in $\mathbb{R}^{3}$ as $\mathbf{C}$. That is to say that

$$\mathbf{C} =\bigcup\big{\{}\chi_{m,n,p}  : m+n+p=0 \big{\}} .$$

$\mathbf{C}$ now resembles the structure in (FIG). Before describing the transformation that takes $\mathbf{C}$ to its flattened form, it would be best to describe the relations on a subset of the plane that preserves the identification of edges. We will call this surface $\mathbf{U}'$. However, the flattening of the surface takes $\mathbf{C}$ to a translation of $\mathbf{U}'$, we call $\mathbf{U}$ (See FIG). We will denote the metric preserving map from $\mathbf{C}$ to $\mathbf{U}$, by $\Psi$. Defining these edge identifying relations over $\mathbf{U}'$ is more convenient for us than defining them over $\mathbf{U}$, but defining $\Psi$ is more convenient when the map is taken to $\mathbf{U}$. Later on we will use $\mathbf{U}'$ when taking its quotient space.

Now we will describe exactly how to construct $\mathbf{U}'$. Take the subset of the real plane, $\mathbf{P} = \mathbb{R}^{2}-\{(u,v)|2m-\frac{1}{2}<u<2m+\frac{1}{2}, 2n-\frac{1}{2}<v<2n+\frac{1}{2} \}$, with 1$\times$1 open squares cut out at every even integer pair and let $\mathbf{U}'$ represent the flattened structure with the proper edge identifications with the first "hole" centered at the origin. We will describe this surface as a product of the auxiliary plane,$z=0$ and the image of a natural map from $\mathbf{P}$ to its quotient space with the following relation on $\mathbf{P}$:

\begin{gather*}
\text{Let } x_{0}=(u_{0},v_{0}),x_{1}=(u_{1},v_{1}) \in \mathbf{P}. \text{Let } \mathbf{R} \text { be a relation described as } x_{0}\mathbf{R}x_{1} \\ \text{ iff } x_{0}=x_{1}
 \text{ or if }x_{0},x_{1} \in {\partial} \left( \left[2m-\frac{1}{2},2m+\frac{1}{2}\right] \times \left[2n-\frac{1}{2},2n+\frac{1}{2}\right] \right)\\
  \left[\begin{array}{c}
u_{1} -2m
\\v_{1}-2n
\end{array}\right] = \left[\begin{matrix}
0 && 1\\
1 && 0
\end{matrix}\right]
\left[ \begin{array}{c}u_{0}-2m\\
v_{0}-2n
\end{array}\right]
\\\text{For some }m,n\in\mathbb{Z} \end{gather*}

The relation will either relate every point to itself, or (in the case where any two points lie on the boundary of the square shaped holes) take the center of each square shaped hole in the plane to the origin and reflect it over the line $y=x$ to match up these points accordingly.  Let $N: \mathbf{P} \rightarrow \mathbf{P}/\mathbf{R}$ denote the natural map from every point on $\mathbf{P}$ to its appropriate equivalence class. Now, let $\mathbf{U}' = Im N \text{ }\times\{0\}$ be the flat representation of the surface that has been translated so as to center the first square cut-out region over the origin. The map $\mathbf{U}\rightarrow\mathbf{U}'$ is given by an affine translation of the surface $(x,y,z) \mapsto (x-\frac{3}{2},y-\frac{3}{2},z)$. (The point at the origin of $\mathbf{C}$ is mapped to $(-\frac{3}{2},-\frac{3}{2},0) \text{ on } \mathbf{U}'$.)

\subsection{Flattening $\mathbf{C}$}
This section will describe the isomorphic map $\Psi:\mathbf{C}\rightarrow\mathbf{U}$, its inverse, $\Psi^{-1}:\mathbf{U}\rightarrow\mathbf{C}$, and prove that $\mathbf{C}$ and $\mathbf{U}$ are isometric.

$\Psi$ is defined piecewise, and denote each subset of $\mathbf{C}$ with the letters $A$, $B$, $C$. We do the same for $\mathbf{U}$ but use instead $A'$, $B'$, $C'$. 

\begin{gather*}
	\text{Let }x,y,z\in\mathbb{R}. \text{ Then}
	\\A=\bigcup\big\{\mathbf{C}\cap\{(x,y,p)\in\mathbb{R}\}: p\in\mathbb{Z}\big\}
	\\B=\bigcup\big\{\mathbf{C}\cap\{(x,n,z)\in\mathbb{R}\}: n\in\mathbb{Z}\big\}
	\\C=\bigcup\big\{\mathbf{C}\cap\{(m,y,z)\in\mathbb{R}\}: m\in\mathbb{Z}\big\}
	\\\text{are the collection of the faces of cubes that are parallel to eachother.}
\end{gather*}

To describe the subsets $A'$, $B'$, and $C'$ properly we will introduce the function $\mu_{n}:\mathbb{R}\rightarrow \left[0,n\right) $ as shorthand for the family of functions that take a real number to its residue class of reals modulo n, for some integer n.

\begin{gather*}
	\text{Let }u,v\in\mathbb{R}. \text{Then }
	\\A'= \{(u,v,0): 0<\mu_{2}(u)\leq 1 ,\hspace{4mm} 0<\mu_{2}(v)\leq 1\}
	\\B'= \{(u,v,0): 1<\mu_{2}(u)\leq 2,\hspace{4mm} 0<\mu_{2}(v)\leq 1\}
	\\C'= \{(u,v,0): 0<\mu_{2}(u)\leq 1,\hspace{4mm} 1<\mu_{2}(v)\leq 2\}
	\\\text{are the collections of the associated faces of the flattened structure.}
\end{gather*}

Now we will define $\Psi$ piece-wise.

\begin{equation}
\Psi\left[\begin{array}{c}
	x\\y\\z
\end{array}\right] 
= 
\begin{cases}
	\left[ \hspace{2mm} \begin{matrix}
		1 & 0 & 0 \\
		0 & 1 & 0 \\
		0 & 0 & 1
	\end{matrix}\hspace{3mm}\right]

	\left[\begin{array}{c}
	x - \left\lfloor x \right\rfloor
	\\ y- \left\lfloor y \right\rfloor
	\\ -(z - \left\lfloor z \right\rfloor)
	\end{array} \right]
	+
	\left[\begin{array}{c}
		2 \left\lfloor x \right\rfloor
		\\ 2\left\lfloor y \right\rfloor
		\\ z - \left\lfloor z \right\rfloor
	\end{array} \right]
		& \text{if } (x,y,z)\in A	\vspace{2mm}
	\\
		
		
	\left[ \begin{matrix}
	0 & 0 & 1 \\
	0 & 1 & 0 \\
	-1 & 0 & 0
	\end{matrix}\hspace{2mm}\right]
	\left[\begin{array}{c}
		x - \left\lfloor x \right\rfloor
		\\ y- \left\lfloor y \right\rfloor
		\\ -(z - \left\lfloor z \right\rfloor)
		\end{array} \right]
	+
		\left[\begin{array}{c}
			2 \left\lfloor x \right\rfloor
			\\ 2\left\lfloor y \right\rfloor
			\\ x - \left\lfloor x \right\rfloor
		\end{array} \right]
			& \text{if } (x,y,z)\in B	\vspace{2mm}
	\\
	
		\left[ \begin{matrix}
		1 & 0 & 0 \\
		0 & 0 & 1 \\
		0 & -1 & 0
		\end{matrix}\hspace{2mm}\right]
		\left[\begin{array}{c}
			x - \left\lfloor x \right\rfloor
			\\ y- \left\lfloor y \right\rfloor
			\\ -(z - \left\lfloor z \right\rfloor)
			\end{array} \right]
		+
			\left[\begin{array}{c}
				2 \left\lfloor x \right\rfloor
				\\ 2\left\lfloor y \right\rfloor
				\\ y - \left\lfloor y \right\rfloor
			\end{array} \right]
				& \text{if } (x,y,z)\in C	\vspace{2mm}
\end{cases}
\end{equation}

We chose to preserve the matrix structure in order to emphasize the manner in which every square is taken to $\mathbf{U}$ from $\mathbf{C}$ and vice-versa. The negations of some of these variables are a result of this structure being built so as the integer triples lie on the plane $z=-(x+y)$. Defining the maps this way also makes it possible to generalize it for any surface constructed in a similar manner of taking unions of three faces of rectangular prisms in real space. (i.e. $\chi_{m,n,p}$ belongs to a larger family of rectangular arrangements parameterized by the dimensions of the rectangular prism. In this case, $\chi_{m,n,p}$ is a member of this family whose dimensions are $1\times1\times1$). 

Now we will define the inverse map, $\Psi^{-1}$, in a similar manner. 

\begin{equation}
	\Psi^{-1}\left[\begin{array}{c}
		x\\y\\z
	\end{array}\right] 
	= 
	\begin{cases}
		\left[ \hspace{2mm} \begin{matrix}
			1 & 0 & 0 \\
			0 & 1 & 0 \\
			0 & 0 & 1
		\end{matrix}\hspace{3mm}\right]
	
		\left[\begin{array}{c}
		x - 2\left\lfloor \frac{x}{2} \right\rfloor
		\\ y- 2\left\lfloor \frac{y}{2} \right\rfloor
		\\ z
		\end{array} \right]
		+
		\left[\begin{array}{c}
			\left\lfloor \frac{x}{2} \right\rfloor
			\\ \left\lfloor \frac{y}{2} \right\rfloor
			\\ -(\left\lfloor \frac{x}{2} \right\rfloor + \left\lfloor \frac{y}{2} \right\rfloor )
		\end{array} \right]
			& \text{if } (x,y,z)\in A'	\vspace{2mm}
		\\
		\left[ \hspace{2mm} \begin{matrix}
			0 & 0 & -1 \\
			0 & 1 & 0 \\
			1 & 0 & 0
		\end{matrix}\hspace{0mm}\right]
	
		\left[\begin{array}{c}
		-(x - 2\left\lfloor \frac{x}{2} \right\rfloor)
		\\ y- 2\left\lfloor \frac{y}{2} \right\rfloor
		\\ z
		\end{array} \right]
		+
		\left[\begin{array}{c}
			1+\left\lfloor \frac{x}{2} \right\rfloor
			\\ \left\lfloor \frac{y}{2} \right\rfloor
			\\ -(\left\lfloor \frac{x}{2} \right\rfloor + \left\lfloor \frac{y}{2} \right\rfloor -1)
		\end{array} \right]
			& \text{if } (x,y,z)\in B'	\vspace{2mm}

	\\
	
		\left[ \hspace{2mm} \begin{matrix}
			1 & 0 & 0 \\
			0 & 0 & -1 \\
			0 & 1 & 0
		\end{matrix}\hspace{0mm}\right]
	
		\left[\begin{array}{c}
		x - 2\left\lfloor \frac{x}{2} \right\rfloor
		\\ -(y- 2\left\lfloor \frac{y}{2} \right\rfloor)
		\\ z
		\end{array} \right]
		+
		\left[\begin{array}{c}
			\left\lfloor \frac{x}{2} \right\rfloor
			\\1+ \left\lfloor \frac{y}{2} \right\rfloor
			\\ -(\left\lfloor \frac{x}{2} \right\rfloor + \left\lfloor \frac{y}{2} \right\rfloor -1)
		\end{array} \right]
			& \text{if } (x,y,z)\in B'	\vspace{2mm}
			
	\end{cases}
\end{equation}

By using properties of the floor function, one can show that $\Psi$ is indeed a bijective, invertible map. All that is left to show is that $\Psi$ preserves distance.

\begin{thm}{$\mathbf{C}$ and $\mathbf{U}$ are isometric.}\\
Proof. Let $(\mathbf{C}, d)$ and $(\mathbf{U}, d')$ be topological spaces with their respective metrics. Let $a,b\in\mathbf{C}$, and let $\Psi(a)=a',\Psi(b)=b'\in\mathbf{U}$. We will denote the distance between $a$ and $b$ by $D=d(a,b)$, and the distance between $a'$ and $b'$ by $D'=d'(a',b')$. Since $\Psi$ and $\Psi^{-1}$ are linear transformations that act on subsets of auxiliary planes in $\mathbb{R}^{3}$, the images of any curves on $\mathbf{C}$ or $\mathbf{U}$ under these maps will not be deformed and retain their lengths. Thus, $D \leq d'(\Psi(a), \Psi(b)) = d'(a',b') = D'$. Likewise, $D' \leq d(\Psi^{-1}(a'),\Psi^{-1}(b')) = d(a,b) = D$. Therefore $D=D'$, and $\Psi$ and $\Psi^{-1}$ are metric preserving maps between $\mathbf{C}$ and $\mathbf{U}$. $\qed$
\end{thm}

As a consequence of this theorem, any dynamical properties of geodesic flow on the surface $\mathbf{U}'$ immediately applies to $\mathbf{C}$. ($\mathbf{U}'$ and $\mathbf{U}$ are trivially isometric).

\section{The Four-Fold Cover and Translation Surface}


\end{document}
